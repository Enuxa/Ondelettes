%!TEX program = PdfLatex
\documentclass[]{article}
\usepackage[utf8]{inputenc}
\usepackage{pdfpages}
\usepackage{amsmath}
\usepackage{amssymb}
\usepackage{graphicx}
\usepackage{geometry}
\usepackage{enumitem}
\usepackage{amsthm}
\usepackage{stmaryrd}
\usepackage{mathtools}
\usepackage{mathrsfs}
\usepackage{bbm}

\geometry{hmargin=2cm}

% Environnement type théorème
\newtheorem{mythm}{Théorème}
\newtheorem{myproposition}{Proposition}
\newtheorem{myproperty}{Propriété}
\newtheorem{mylemma}{Lemme}
\newtheorem{mycoro}{Corollaire}

% Environnement type texte
\theoremstyle{remark}
\newtheorem{mynot}{Notation}
\newtheorem{myrem}{Remarque}
\newtheorem{myexer}{Exercice}
\newtheorem{myproof}{Preuve}
\newtheorem{myexmpl}{Exemple}

% Environnement de définition
\theoremstyle{definition}
\newtheorem{mydef}{Définition}
\newtheorem{myquestion}{Question}

\setlist[itemize]{label=-}

% Carré de fin de preuve
\newcommand{\cqfd}{
	\hfill$\square$
}

% Définition de fonction
\newcommand{\func}[5]{
#1 ~ : ~ \left\{ \begin{array}{lcl}
	#2 & \longrightarrow & #3 \\
	#4 & \longmapsto & #5
\end{array}
\right.
}

\newcommand{\fun}[3]{
#1 ~ : ~ #2 \longrightarrow #3
}

\newcommand{\funcinline}[5]{
	#1 \, : \, #2 \longrightarrow #3, ~ #4 \longmapsto #5
}

\newcommand{\funcshort}[3]{
	#1 \, : \, #2 \longrightarrow #3
}

\newcommand{\anonfunc}[4]{
	\left\{ \begin{array}{lcl}
		#1 & \longrightarrow & #2 \\
		#3 & \longmapsto & #4
	\end{array}
	\right.
}

\newcommand{\vect}{\text{Vect}}

\newcommand{\card}{\text{Card }}

\newcommand{\DS}{\displaystyle}

\setlength\parindent{0pt}

\begin{document}

\part{Algorithmes d'encodage et de décodage}

\section{Cadre}

Dans cette partie, nous présenterons un algorithme pour encoder une fonction $f:\llbracket0,N-1\rrbracket\to[-1,1]$ quelconque suivant une analyse multi-résolution.

On se place dans une analyse multi-résolution de $L^2(\mathbb R)$ notée $(V_n)_{n\in\mathbb Z}$ engendrée par $\varphi$.

%Les espaces de détails $W_{-\lceil\log_2 N\rceil} \stackrel{\perp}{\oplus}\cdots\stackrel{\perp}{\oplus} W_0$ suffiront pour encoder $f$.

On cherche donc un algorithme pour calculer les coordonnées de $f$ dans l'espace :
$$W_{-\lceil\log_2 N\rceil} \stackrel{\perp}{\oplus}\cdots\stackrel{\perp}{\oplus} W_0$$
On rappelle que :

\begin{flalign*}
f &= \sum_{n\in\mathbb Z} \sum_{k\in\mathbb Z} \langle f,\psi_{n,k}\rangle \psi_{n,k} = \sum_{n\in\mathbb Z} \sum_{k\in\mathbb Z} \left(\int_0^N f(t) \overline{\psi_{n,k}(t)} dt\right) \psi_{n,k} \\
&= \sum_{n\in\mathbb Z} \sum_{k\in\mathbb Z} \left(\sum_{x=0}^{N-1}f(x)\int_x^{x+1} \overline{\psi_{n,k}(t)} dt\right) \psi_{n,k}
\end{flalign*}


\end{document}