%!TEX program = PdfLatex
\documentclass[]{article}
\usepackage[utf8]{inputenc}
\usepackage{pdfpages}
\usepackage{amsmath}
\usepackage{amssymb}
\usepackage{graphicx}
\usepackage{geometry}
\usepackage{enumitem}
\usepackage{amsthm}
\usepackage{stmaryrd}
\usepackage{mathtools}
\usepackage{mathrsfs}
\usepackage{bbm}

\geometry{hmargin=2cm}

% Environnement type théorème
\newtheorem{mythm}{Théorème}
\newtheorem{myproposition}{Proposition}
\newtheorem{myproperty}{Propriété}
\newtheorem{mylemma}{Lemme}
\newtheorem{mycoro}{Corollaire}

% Environnement type texte
\theoremstyle{remark}
\newtheorem{mynot}{Notation}
\newtheorem{myrem}{Remarque}
\newtheorem{myexer}{Exercice}
\newtheorem{myproof}{Preuve}
\newtheorem{myexmpl}{Exemple}

% Environnement de définition
\theoremstyle{definition}
\newtheorem{mydef}{Définition}
\newtheorem{myquestion}{Question}

\setlist[itemize]{label=-}

% Carré de fin de preuve
\newcommand{\cqfd}{
	\hfill$\square$
}

% Définition de fonction
\newcommand{\func}[5]{
#1 ~ : ~ \left\{ \begin{array}{lcl}
	#2 & \longrightarrow & #3 \\
	#4 & \longmapsto & #5
\end{array}
\right.
}

\newcommand{\fun}[3]{
#1 ~ : ~ #2 \longrightarrow #3
}

\newcommand{\funcinline}[5]{
	#1 \, : \, #2 \longrightarrow #3, ~ #4 \longmapsto #5
}

\newcommand{\funcshort}[3]{
	#1 \, : \, #2 \longrightarrow #3
}

\newcommand{\anonfunc}[4]{
	\left\{ \begin{array}{lcl}
		#1 & \longrightarrow & #2 \\
		#3 & \longmapsto & #4
	\end{array}
	\right.
}

\newcommand{\vect}{\text{Vect}}

\newcommand{\card}{\text{Card }}

\newcommand{\DS}{\displaystyle}

\setlength\parindent{0pt}

\begin{document}

\part{Algorithmes d'encodage et de décodage}

\section{Algorithme d'encodage en 1 dimension}

\subsection{Cadre}

Dans cette partie, on veut encoder une fonction $f:\llbracket0,N-1\rrbracket\to[-1,1]$ quelconque.

On se place dans une analyse multi-résolution de $L^2(\mathbb R)$ notée $(V_n)_{n\in\mathbb Z}$ engendrée par $\varphi$.

On rappelle que $f$ peut être recalculé à partir de ses coordonnées $\langle f,\psi_{n,k}\rangle$ :

\begin{flalign*}
f &= \sum_{n\in\mathbb Z} \sum_{k\in\mathbb Z} \langle f,\psi_{n,k}\rangle \psi_{n,k}
\end{flalign*}

où :

\begin{flalign*}
\langle f,\psi_{n,k}\rangle
&= \int_0^N f(t) \overline{\psi_{n,k}(t)} dt \\
%&= \sum_{n\in\mathbb Z} \sum_{k\in\mathbb Z} \left(\sum_{x=0}^{N-1}f(x)\int_x^{x+1} \overline{\psi_{n,k}(t)} dt\right) \psi_{n,k} \\
&= \sum_{x=0}^{N-1}f(x)\int_x^{x+1} \overline{\psi(2^{-n} t-k)} dt \\
&= 2^{-n}\sum_{x=0}^{N-1}f(x)[F(t)]_{2^{-n}x-k}^{2^{-n}(x+1)-k} \\
\end{flalign*}

où $F$ est une primitive de $\overline\psi$ .

\subsection{Suppression des cordonnées inutiles}

\subsubsection{Coordonnées $n\le 0$}

La valeur de $\langle f,\psi_{n,k}\rangle$ représente à peu près la variation dans l'intervalle $[2^{n}k,2^{n}(k+1)]$.

Étant donné que $f$ est échantillonné avec une fréquence de $1$, $f$ est constant sur les intervalles $[2^{n}k,2^{n}(k+1)]$ pour $n\le 0$. On néglige donc la valeur des coordonnées à ces indices.

\subsubsection{Coordonnées $n > \lceil\log_2 N\rceil$}

\subsubsection{Coordonnées $k$}

On veut obtenir un recouvrement minimal de $[0,N[$ avec les intervalles disjoints $[2^{n}k,2^{n}(k+1)[$ .

Pour que $[2^{n}k,2^{n}(k+1)[\cap[0,N[\neq\emptyset$, il faut que $0\le k< 2^{-n}N$ .

\subsection{Code de l'algorithme}

\subsection{Exemples}









\end{document}