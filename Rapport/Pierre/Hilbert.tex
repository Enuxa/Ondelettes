	\newpage
	\part{Outils préliminaires}
	\section{Espaces de Hilbert}
	\subsection{Introduction et définition}

	\paragraph*{}
	
	Nous disposons en dimension finie de nombreux théorèmes bien utiles, notamment dans le cadre des espaces euclidiens (existence d'une base, toute famille orthogonale est libre, procédé d'orthonormalisation de Gramm-Schimdt ...).
	
	Cela se prête parfaitement à des études géométriques dans le plan et l'espace que nous connaissons, mais également à la manipulation de certains objets de l'analyse tels que les polynômes de degré au plus $n \in \mathbb{N}$ ou de matrices que l'on identifie à certaines applications linéaires.
	
	\paragraph*{}
	
	Rappelons rapidement la notion de base :
	
	\begin{mydef}
		pour un espace vectoriel $E$, une \textit{base} $\mathcal{B} \subset E$ est une famille \textit{libre} et \textit{génératrice} : tout élément de $E$ peut s'écrire comme une unique combinaison linéaire (finie) d'éléments de $\mathcal{B}$. Les bases de $E$ sont toutes de même cardinal, on définit grâce à elles la \textit{dimension} de $E$ comme étant le cardinal de $\mathcal{B}$.
	\end{mydef}
	
	L'axiome du choix nous garanti que tout espace vectoriel admet une base. L'espace des suites finies à partir d'un certain rang (que l'on notera $\ell_F$), bien que de dimension infinie, possède une base très simple : $$\mathcal{U} = \{\nu_1 = (1, 0, 0 \cdots), \nu_2 = (0, 1, 0, \cdots), \nu_3 = (0, 0, 1, 0 \cdots), \nu_4 = (0, 0, 0, 1, 0 \cdots), \cdots\}$$
	
	En effet toute suite de $\ell_F$ est une combinaison linéaire (finie) de tels éléments. Cependant cette famille ne constitue pas une base pour l'espace $\ell^1$ des suites de module sommable et on souhaite utiliser la famille $\mathcal{U}$ qui est très naturelle et facile à manipuler.
	
	Nous sommes très tentés d'écrire les éléments de $\ell_F$ comme une somme infinie d'éléments de $\mathcal{U}$, mais pour que cela ait un sens il est nécessaire de définir une topologie, par exemple avec une norme, voire une norme induite par un produit scalaire.
	
	\paragraph*{}
	
	La question se pose alors ; pouvons nous généraliser les outils de géométrie pour étudier des espaces de dimension infinie ?
	
	C'est dans ce contexte qu'est née l'analyse de Hilbert au début du 20-ième siècle au travers des travaux de Erhard Schimdt, Frigyes Riesz et bien sûr David Hilbert.
	
	\subsection{Bases de Hilbert}
	
	\begin{mydef}
		Un espace vectoriel $E$ est dit \textit{de Hilbert} (ou \textit{Hilbertien}) s'il est muni d'un produit scalaire $\langle \cdot, \cdot \rangle$ et qu'il est complet pour la norme induite par ce produit scalaire.
	\end{mydef}
	
	On ne s'intéressera ici qu'à un certain type d'espaces hilbertiens ; ceux qui sont dits \textit{séparables} :
	
	\begin{mydef}
		Un espace vectoriel normé $E$ est dit \textit{séparable} s'il admet une famille \textit{dense} et \textit{dénombrable}.
		
		C'est-à-dire s'il existe $F = \{f_n\}_{n \in \mathbb{N}} \subset E$ telle que $\overline{F} = E$
	\end{mydef}
	
	\begin{myexmpl}
		\leavevmode
		\begin{enumerate}
			\item $\mathbb{R}$ muni de la valeur absolue est séparable : $\overline{\mathbb{Q}} = \mathbb{R}$, et l'ensemble des rationnels est bien dénombrable.
			
			\item Plus généralement si $E \cong \mathbb{R}^n$ pour un certain $n \geqslant 1$, et $\mathbf{e}=(e_i)_{1 \leqslant i \leqslant n}$ est une base de $E$, alors on construit l'ensemble des combinaisons linéaires de $\mathbf{e}$ à coefficients dans $\mathbb{Q}$ :
			
			$$F = \left\{\DS \sum_{i = 1}^{n} q_i e_i ~ | ~ \{q_i\}_i \subset \mathbb{Q} \right\}$$
			
			Cet ensemble est bien dénombrable : il est équipotent à $\mathbb{Q}^n$, et il s'agit bien d'une partie dense dans $E$.
			
			En effet pour tout $\DS x = \sum_{i = 1}^n x_i e_i$ on peut trouver $n$ suites $\{q_{i, j}\}_{j \in \mathbb{N}} \subset \mathbb{Q}$ telles que $q_{i, j} \to x_i ~ (j \to \infty)$ pour ainsi obtenir $\DS x_j := \sum_{i = 1}^{n} q_{i, j} e_i$ de limite $x$.
		\end{enumerate}
	\end{myexmpl}
	
	\paragraph*{}
	
	On remarque que si on s'autorisait à prendre dans le second exemple des coefficients réels on aurait $F = E$ car $\mathbf{e}$ est une base. Dans un cas plus général (ou plutôt un cas bien particulier ; toujours celui de $\ell^1$), si $\mathbf{e}$ n'est pas une base mais seulement une famille libre agréable à manipuler et que l'on souhaite absolument utiliser, est-ce que l'ensemble des combinaisons linéaires d'éléments de $e$ donne tout l'espace initial ? (Non, sinon j'aurais précisé qu'elle était également génératrice).
		
	\begin{mydef}
		Une partie $P \subset E$ est dite totale si $Vect(P) = \{\text{combinaisons linéaires d'éléments de $P$}\}$ est dense dans $E$.
	\end{mydef}
	
	\begin{myexmpl}
		Prenons l'exemple de $\ell^2$ muni de la norme suivante $$\|u\|_2 = \sqrt{\sum_{n = 0}^\infty |u_n|^2}$$
		
		Montrons que toute suite de cet espace s'écrit dans $\mathcal{U}$ de la forme $\DS u = \sum_{n = 0}^\infty u_n \nu_n$, où la convergence a lieu au sens de $\|\cdot\|_2$.
		
		Soit $u \in \ell^2$, on a $$\left\|u - \sum_{k = 0}^{n} \nu_k u_k\right\|_2^2 = \left\|(0, 0, \cdots u_{n+1}, u_{n+2} \cdots)\right\|_2^2 = \sum_{k = n+1}^\infty |u_k|^2$$
		
		Or $\sum |u_n|^2$ est une suite convergente, la valeur précédente converge donc bien vers 0. Toute suite de $\ell^2$ peut alors être écrite comme une limite de combinaisons linéaires d'éléments de $\mathcal{U}$, $Vect(\mathcal{U})$ est bien une partie totale de $\ell^2$.
	\end{myexmpl}
	
	\paragraph*{}
	La famille $\mathcal{U}$ a beau ne pas être une base, elle a le mérite d'être facile à appréhender et de permettre de décomposer tout élément de $\ell^2$ en une série absolument convergente. On munit à présent cet espace du produit scalaire induisant $\|\cdot\|_2$ : $$\langle u, v \rangle = \sum_{n = 0}^\infty u_n \overline{v_n}$$
	On remarque que $\mathcal{U}$ est orthonormale pour ce produit scalaire.
	
	\begin{myproposition}
		$\ell^2$ muni de ce produit scalaire forme un espace Hilbertien.
	\end{myproposition}

	\begin{myproof}
		En effet soit $(u_n)_{n \in \mathbb{N}}$ une suite de Cauchy à valeurs dans $\ell^2$, on a pour tout $\varepsilon > 0$ un entier $N \geqslant 0$ tel que pour tout $p \geqslant 0$ $$\|u_N - u_{N+p}\|^2_2 = \sum_{n=0}^{\infty} |u_{N, n} - u_{N+p, n}|^2 \leqslant \varepsilon$$
		en notant $u_N=(u_{N, 0}, u_{N, 1}, u_{N, 2} \cdots)$.
		
		On en déduit pour tout $n$ $|u_{N, n} - u_{N+p, n}|^2 \leqslant \varepsilon$ ce qui signifie que chaque suite $(u_{k, n})_{k \in \mathbb{N}}$ est une suite de Cauchy à valeurs dans $\mathbb{C}$ qui lui est complet.
		
		Soient $v_n$ la limite des $n$-ième composante des suites $(u_N)_{N \in \mathbb{N}}$, on a pour tout $M$ :
		
		$$\sum_{n=0}^{M} |v_n|^2 = \lim\limits_{N \to \infty} \sum_{n=0}^{M} |u_{N, n}|^2 \leqslant \lim\limits_{N\to \infty} \|u_N\|_2^2$$
		
		Or la suite $(u_N)_N$ est de Cauchy donc bornée, la somme est alors elle aussi bornée donc convergente car à termes positifs. $(v_n)_n$ est donc bien un élément de $\ell^2$.
		
		\cqfd
	\end{myproof}
	
	$\mathcal{U}$ est alors une famille orthonormée et totale de $\ell^2$, c'est-à-dire "presque" génératrice (si on accepte d'abandonner la restriction de combinaisons linéaires finies), cela mérite tout de même un nom !
	
	\begin{mydef}
		Une famille $\mathcal{F} \subset H$ est appelée \textit{base hilbertienne} si elle est orthonormale et totale.
	\end{mydef}
	
	Cette notion est une généralisation de celle de base, ce qui soulève immédiatement certaines questions :
	
	\begin{itemize}
		\item Existe-il toujours une base hilbertienne ?
		\item Les bases Hilbertiennes sont elles toutes de même cardinal ?
		\item Toute base hilbertienne est elle encore une famille orthonormale maximale pour l'inclusion et réciproquement ?
	\end{itemize}
	
	Nous avons annoncé plus haut que nous ne nous intéresserions qu'aux espaces séparables, ce qui nous permet de répondre à la première question :
	
	\begin{myproposition}
		Tout espace Hilbertien séparable admet une base Hilbertienne.
	\end{myproposition}
	
	\begin{myproof}
		Soit $E$ un espace hilbertien et $\mathbf{g} = \{g_n\}_{n \in \mathbb{N}} \subset E$ une partie dense dans $E$. Nous allons construire une base hilbertienne $\mathbf{f} = \{f_n\}_{n \in \mathbb{N}}$ à partir de $\mathbf{g}$ en l'orthonormalisant par le procédé de Gramm-Schmidt.
		
		On supposera $0 \notin \mathbf{g}$ et $E$ de dimension infinie, dans le cas de la dimension finie la suite construite par récurrence est finie et est une base en tant que famille libre (car orthonormée) maximale.
		
		
		On note $G_n = \vect (g_0, g_1 \cdots g_n)$ et $F_n = \vect (f_0, f_1 \cdots f_n)$.
		
		\paragraph{Construction et orthonormalité}
		
		On pose $\DS f_0 = \frac{g_0}{\|g_0\|}$ et on construit par récurrence la suite $\mathbf{f}$ de manière à ce qu'à tout rang $n \in \mathbb{N}$ la famille $(f_0, f_1 \cdots f_n)$ soit orthonormale.
		
		Soit $n \geqslant 0$, on suppose disposer de $f_0, f_1 \cdots f_n$ comme ci-dessus. $E$ est de dimension infinie et $\mathbf{g}$ est dense, il existe alors un plus petit $m$ tel que $G_m \setminus F_n \neq \emptyset$. On choisit un $x \in G_m \setminus F_n$, celui-ci est alors linéairement indépendant des vecteurs $f_0$ à $f_n$, on peut donc  l'orthonormaliser :
		
		$$y = x - \sum_{k = 0}^{n} \langle x, f_k \rangle f_k$$
		
		$$f_{n+1} = \frac{y}{\|y\|}$$
		
		$(f_0, f_1 \cdots f_{n+1})$ est ainsi une famille orthonormale.
		
		\paragraph{Totalité}
		
		A tout rang $n$ on a $\{g_0, \cdots g_n\} \subset F_n$ et donc $G_n \subset F_n$. En faisant l'union pour tout $n \in \mathbb{N}$ on obtient $\DS \mathbf{g} \subset \bigcup_{n \in \mathbb{N}} G_n \subset \bigcup_{n \in \mathbb{N}} F_n$. $\mathbf{g}$ étant par hypothèse dense dans $E$, on obtient le résultat en passant à l'adhérence :
		
		$$E = \overline{\mathbf{g}} \subset \overline{\bigcup_{n \in \mathbb{N}} F_n} = \overline{\vect (f_0, f_1 \cdots)} \subset E$$
		\cqfd
	\end{myproof}
	
	Ainsi il existe toujours une base hilbertienne dans un espace hilbertien séparable. De plus, toutes les bases hilbertiennes sont équipotentes entre elles.
	
	Cette propriété est évidente dans le cas de la dimension finie, pour le démontrer en dimension infinie, il suffit de montrer que toute base s'injecte dans $\mathbb{N}$ où plus généralement dans un ensemble dénombrable.
	
	\begin{myproposition}
		Dans un espace hilbertien séparable, toutes les bases sont de même cardinal.
	\end{myproposition}
	
	\begin{myproof}
		Soit $E$ un espace de Hilbert séparable de dimension infinie et $\mathcal{B} = \{f_i ~ | ~ i \in I\}$ une base de $E$.
		
		Par définition $E$ admet une famille dense dénombrable $\mathcal{F} = \{g_n ~ | n \in \mathbb{N}\}$, nous allons approximer notre base par ses éléments ce qui donnera une injection de $\mathcal{B}$ dans $\mathcal{F}$.
		
		Soit $\varepsilon > 0$, pour tout $f \in \mathcal{B}$, il existe un $g \in \mathcal{F}$ à distance au plus $\varepsilon$ de $f$, on pose $\varphi_\varepsilon(f) = g$.
		
		Ainsi pour tout $f, f' \in \mathcal{B}$ on a l'inégalité suivante :
		
		$$\|f - f'\| \leqslant \|f - \varphi_\varepsilon(f)\| + \|\varphi_\varepsilon(f) - \varphi_\varepsilon(f')\| + \|f' - \varphi_\varepsilon(f')\| \leqslant 2 \varepsilon + \|\varphi_\varepsilon(f) - \varphi_\varepsilon(f')\|$$
		
		$$\|f - f'\| - 2 \varepsilon \leqslant \|\varphi_\varepsilon(f) - \varphi_\varepsilon(f')\|$$
		
		$\mathcal{B}$ étant orthonormée, si $f$ et $f'$ sont distincts, la valeur $\|f-f'\| = \sqrt{2}$ est indépendante de $f$, $f'$ et $\varepsilon$, ainsi pour $\varepsilon$ assez petit $$\forall f, f' \in E, ~ f \neq f' \Longrightarrow \|\varphi_\varepsilon(f) - \varphi_\varepsilon(f')\| > 0$$ c'est-à-dire $\funcshort{\varphi_\varepsilon}{\mathcal{B}}{\mathcal{F}}$ est injective car on a $x \neq y \Longrightarrow \varphi_\varepsilon(x) \neq \varphi_\varepsilon(y)$.
		
		\cqfd
	\end{myproof}
	
	Répondons à présent à la troisième question : il y a-t-il encore équivalence entre être une base hilbertienne et être une famille orthonormée maximale pour l'inclusion ?
	
	La réponse est oui !
	
	\begin{mythm}
		Dans un espace Hilbertien, une famille orthonormée est une base hilbertienne si et seulement si elle est maximale.
	\end{mythm}
	
	Ici "maximal" est au sens d'une partie vérifiant une certaine propriété (que tous ses éléments soient orthogonaux entre eux), c'est-à-dire qu'elle contient tous les vecteurs non-nuls orthogonaux à tous ses autres éléments.
	
	Pour démontrer ce théorème, nous aurons besoin des lemmes suivants :
	
	\begin{mylemma}
		Soit $\{e_n ~ | ~ n \in \mathbb{N}\}$ une famille orthonormée et $F = \vect \, (e_0, e_1 \cdots)$
		\begin{enumerate}
			\item $\forall n \in \mathbb{N}, ~ \forall f \in E, ~ \DS \sum_{k = 0}^{n} |\langle f, e_k \rangle|^2 \leqslant \|f\|^2$ (Inégalité de Bessel)
			\item L'application $\funcshort{\pi_F}{E}{\overline{F}}$ donnée par $\DS \pi_F(f) = \sum_{n = 0}^\infty \langle f, e_n \rangle e_n$ est bien définie et est la projection orthogonale de $E$ sur $\overline{F}$.
		\end{enumerate}
	\end{mylemma}
	
	\begin{myproof}
		Preuve de 1. :
		
		Soit $f \in E$ et $n \in \mathbb{N}$
		
		$$\left\| f - \sum_{k = 0}^{n} \langle f, e_k \rangle e_k \right\|^2 \geqslant 0$$
		
		$$\|f\|^2 + \left\| \sum_{k = 0}^{n} \langle f, e_k\rangle e_k \right\|^2 - 2 Re \left \langle f , \sum_{k = 0}^{n} \langle f, e_k \rangle e_k \right\rangle \geqslant 0$$
		
		$$\|f\|^2 + \sum_{k = 0}^{n} | \langle f, e_k\rangle | ^2 - 2 Re \left \langle f , \sum_{k = 0}^{n} \langle f, e_k \rangle e_k \right\rangle \geqslant 0$$
		
		$$\|f\|^2 + \sum_{k = 0}^{n} | \langle f, e_k\rangle | ^2 - 2 Re \sum_{k = 0}^{n} |\langle f , e_k \rangle |^2 \geqslant 0$$
		
		$$\|f\|^2 + \sum_{k = 0}^{n} | \langle f, e_k\rangle | ^2 - 2 \sum_{k = 0}^{n} |\langle f , e_k \rangle |^2 \geqslant 0$$
		
		$$\|f\|^2 \geqslant \sum_{k = 0}^{n} |\langle f , e_k \rangle |^2$$
		
		Preuve de 2.
		
		La série $\DS \sum \langle f, e_n \rangle e_n$ vérifie le critère de Cauchy, or l'espace $E$ est complet en tant qu'espace Hilbertien, alors la série est convergente. En effet pour tout $N, p \geqslant 0$ :
		
		$$\left\|\sum_{k = 0}^{N} \langle f, e_n \rangle e_n - \sum_{k = 0}^{N + p} \langle f , e_n \rangle e_n\right\|^2 = \left\| \sum_{k = N+1}^{N + p} \langle f , e_n \rangle e_n \right\|^2 = \sum_{k = N+1}^{N + p} |\langle f , e_n \rangle|^2$$
		
		Or par le point précédent, la série $\DS \sum | \langle f, e_n \rangle|^2$ est convergente, donc  $\DS \sum_{k = N+1}^{N + p} |\langle f , e_n \rangle|^2$ peut être rendu aussi petit que l'on souhaite, $\pi_F(f)$ existe donc.
		
		Montrons à présent que $f - \pi_F(f)$ est orthogonal à tout vecteur de $F$, c'est-à-dire orthogonal à tout vecteur de la base de $F$.
		
		Soit $e_i$ un vecteur de la famille orthonormée, $f$ un vecteur de $E$
		
		$$
		\begin{aligned}
			\left \langle \sum_{k=0}^n \langle f, e_k \rangle e_k, e_i \right \rangle &= \sum_{k=0}^{n} \langle f, e_k \rangle \langle e_k, e_i \rangle \\
			&= \langle f, e_i \rangle \langle e_i, e_i \rangle \\
			&= \langle f, e_i \rangle
		\end{aligned}
		$$

		Ainsi $\DS \left\langle f - \sum_{k = 0}^n \langle f, e_k \rangle e_k, e_i \right \rangle = \langle f, e_i \rangle - \langle f, e_i \rangle = 0$, on en conclut en faisant tendre $n$ vers $\infty$ que $\pi_F(f) - f \in F^\bot$. $\pi_F$ est bien la projection orthogonale de $E$ sur $\overline{F}$.
		
		\cqfd
	\end{myproof}
	
	Nous pouvons à présent revenir à la preuve du théorème qui nous intéresse.
	
	\begin{myproof}
		Soit $\mathcal{F} = \{e_n ~ | ~ n \in \mathbb{N}\}$ une famille orthonormale d'un espace hilbertien $E$.
		
		\paragraph{Si $\mathcal{F}$ est une base hilbertienne, alors elle est maximale :}
		
		Soit $f$ un vecteur orthogonal à tout élément de $\mathcal{F}$, montrons $f \in \mathcal{F}$.
		
		$\mathcal{F}$ étant une base hilbertienne, pour tout $\varepsilon > 0$ il existe une famille de scalaires $(\lambda)_{i \in I}$ avec $I \subset \mathbb{N}$ fini tel que $$\left\|f - \sum_{i \in I} \lambda_i e_i\right\|^2 < \varepsilon$$
		
		$$\left\langle f - \sum_{i \in I} \lambda_i e_i, f - \sum_{i \in I} \lambda_i e_i \right\rangle < \varepsilon$$
		
		$$\|f\|^2 - \left\langle f, \sum_{i \in I} \lambda_i e_i \right\rangle - \left\langle \sum_{i \in I} \lambda_i e_i, f \right\rangle + \sum_{i \in I} |\lambda_i|^2 < \varepsilon$$
		
		$$\|f\|^2 - \sum_{i \in I} \lambda_i \left\langle f, e_i \right\rangle - \sum_{i \in I} \lambda_i \left\langle e_i, f \right\rangle + \sum_{i \in I} |\lambda_i|^2 < \varepsilon$$
		
		On a supposé $f$ orthogonal à la famille $\mathcal{F}$ :
		
		$$\|f\|^2 + \sum_{i \in I} |\lambda_i|^2 < \varepsilon$$
		
		$$\|f\|^2 < \varepsilon$$
		
		On en déduit $f = 0$, $\mathcal{F}$ est donc maximale.
		
		\paragraph{Si $\mathcal{F}$ est maximale, alors c'est une base hilbertienne :}
		
		D'après le lemme précédent, pour tout $f \in E$, $(f - \pi_F(f)) \in F^\bot$, alors par maximalité de la famille $\mathcal{F}$ on a $f - \pi_F(f) = 0$, d'où le résultat.
		
		\cqfd
	\end{myproof}
	
	\section{Espaces de Lebesgue}
	
		\begin{mydef}
			Étant donnés $\mu$ une mesure sur $\mathbb{R}$ et $p \in [1, +\infty[$, on définit
				$$ \mathcal{L}^{p} (\mathbb{R}) = \left \{ \funcshort{f}{\mathbb{R}}{\mathbb{C}} \text{ mesurable} ~|~ \int_{\mathbb{R}}|f|^p d\mu < \infty \right \} $$
		\end{mydef}
	
		Il s'agit d'un espace vectoriel. De plus, l'application $\funcshort{\| \cdot \|_p}{\mathcal{L}^p(\mathbb{R})}{\mathbb{R}_+}$ donnée par
		 $$\Vert f \Vert_p = \left( \int|f|^pd\mu \right)^{\frac{1}{p}} $$ 
		 définit une semi-norme: elle vérifie bien la propriété d'homogénéité et l'inégalité triangulaire \cite{tmi}, mais pas le fait d'être définie. En effet, certaines fonctions non nulles $f$ vérifient $\|f\|_p = 0$ (si elles sont nulles presque partout). Par exemple,la fonction caractéristique de $\mathbb{Q}$. 
		 
		 On peut palier à ce problème en quotientant $\mathcal{L}^p$ par son sous espace des fonctions nulles presque partout. Si $f$ et $g$ coïncident presque partout, alors
		 $\|f\|_p = \|g\|_p$. Cette application reste donc bien définie sur l'espace quotient, et on a à présent que $ \|[f]\|_p = 0$ implique $f \equiv 0$ presque partout, où $[f]$ est classe d'équivalence de $f$. 
		 Doénavant, on utilisera $f$ pour exprimer la classe d'équivalence de $f$.
		 
		 \begin{mydef} 
		 	On note cet espace quotient $L^p(\mathbb{R})$, et on l'appelle espace de Lebesgue. Muni de la norme $\| \cdot \|_p$, il constitue un espace de Banach (théorème de Riesz-Fisher)\cite{tmi}. 
		 \end{mydef}
		 
		On effectue la même construction pour les fonctions $2\pi$-périodiques que l'on note $L^p(\mathbb{T})$.
		
		\begin{myrem}
			Les fonctions $2\pi$-périodiques peuvent être vues comme des fonctions définies sur le cercle $\mathbb{S}^1$.
		\end{myrem}

		
	
